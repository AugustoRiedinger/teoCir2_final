\message{ !name(../main.tex)}\documentclass[journal,transmag]{IEEEtran}

% Document preamble - loads packages
% *** MISC UTILITY PACKAGES ***
\usepackage[spanish]{babel}
\usepackage{amsmath, amsthm, amssymb, amsfonts} % Nicer mathematical typesetting
\usepackage{lipsum} % Creates dummy text lorem ipsum to showcase typsetting
\usepackage{siunitx}
\usepackage{nicefrac}
\usepackage{widetext}

\usepackage{graphicx} % Allows the use of \begin{figure} and \includegraphics
\usepackage{float} % Useful for specifying the location of a figure ([H] for ex.)
\usepackage{caption} % Adds additional customization for (figure) captions
\usepackage{subcaption} % Needed to create sub-figures
\usepackage{tabularx} % Adds additional customization for tables
\usepackage{tabu} % Adds additional customization for tables
\usepackage{booktabs} % For generally nicer looking tables
\usepackage[nottoc,numbib]{tocbibind} % Automatically adds bibliography to ToC
\usepackage{hyperref} % Allows links and makes references and the ToC clickable
\usepackage[noabbrev, capitalise]{cleveref} % Easier referencing using \cref{<label>} instead of \ref{}

\usepackage{xcolor} % Predefines additional colors and allows user defined colors

% *** GRAPHICS RELATED PACKAGES ***
%
\ifCLASSINFOpdf
  % \usepackage[pdftex]{graphicx}
  % declare the path(s) where your graphic files are
  % \graphicspath{{../pdf/}{../jpeg/}}
  % and their extensions so you won't have to specify these with
  % every instance of \includegraphics
  % \DeclareGraphicsExtensions{.pdf,.jpeg,.png}
\else
  % or other class option (dvipsone, dvipdf, if not using dvips). graphicx
  % will default to the driver specified in the system graphics.cfg if no
  % driver is specified.
  % \usepackage[dvips]{graphicx}
  % declare the path(s) where your graphic files are
  % \graphicspath{{../eps/}}
  % and their extensions so you won't have to specify these with
  % every instance of \includegraphics
  % \DeclareGraphicsExtensions{.eps}
\fi

% Document settings
% correct bad hyphenation here
\hyphenation{op-tical net-works semi-conduc-tor}



\begin{document}

\message{ !name(sections/03-01-ellip.tex) !offset(-11) }
\subsection{Diseño de un filtro elíptico FFIR}
    Se ilustrará a continuación un filtro pasa-bajos de fase lineal FFIR utilizando la técnica de diseño de magnitud cuadarada como se desarrolló en la Sec. \ref{sec:02-02-mag-cuad}. Entonces, se deseará que el filtro cumpla con los siguientes criterios de diseño

    \begin{enumerate}
      \item Banda de paso normalizada $(0.00,0.10)$ en fracciones de $\nicefrac{f_{s}}{2}$ con al menos $0.08 \: [\unit{dB}]$ de ripple máximo pico-a-pico, donde $f_{s}$ se define como la frecuencia de muestreo.
      \item Banda de parada normalzada $(0.11,1.00)$ con al menos $50 \: [\unit{dB}]$ de atenuación.
      \item Fase lineal.
      \item Amplitud máxima de $\mu = 1.0$.
    \end{enumerate}

    Debido a la banda de transición angosta en el intervalo $(0.10,0.11)$, se selecciona un filtro elíptico como la base del diseño.



%%% Local Variables:
%%% mode: latex
%%% TeX-master: "../main"
%%% End:

\message{ !name(../main.tex) !offset(-11) }

\end{document}
