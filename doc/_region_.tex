\message{ !name(../main.tex)}\documentclass[journal,transmag]{IEEEtran}

% Document preamble - loads packages
% *** MISC UTILITY PACKAGES ***
\usepackage[spanish]{babel}
\usepackage{amsmath, amsthm, amssymb, amsfonts} % Nicer mathematical typesetting
\usepackage{lipsum} % Creates dummy text lorem ipsum to showcase typsetting
\usepackage{siunitx}
\usepackage{nicefrac}

\usepackage{graphicx} % Allows the use of \begin{figure} and \includegraphics
\usepackage{float} % Useful for specifying the location of a figure ([H] for ex.)
\usepackage{caption} % Adds additional customization for (figure) captions
\usepackage{subcaption} % Needed to create sub-figures
\usepackage{tabularx} % Adds additional customization for tables
\usepackage{tabu} % Adds additional customization for tables
\usepackage{booktabs} % For generally nicer looking tables
\usepackage[nottoc,numbib]{tocbibind} % Automatically adds bibliography to ToC
\usepackage{hyperref} % Allows links and makes references and the ToC clickable
\usepackage[noabbrev, capitalise]{cleveref} % Easier referencing using \cref{<label>} instead of \ref{}

\usepackage{xcolor} % Predefines additional colors and allows user defined colors

% *** GRAPHICS RELATED PACKAGES ***
%
\ifCLASSINFOpdf
  % \usepackage[pdftex]{graphicx}
  % declare the path(s) where your graphic files are
  % \graphicspath{{../pdf/}{../jpeg/}}
  % and their extensions so you won't have to specify these with
  % every instance of \includegraphics
  % \DeclareGraphicsExtensions{.pdf,.jpeg,.png}
\else
  % or other class option (dvipsone, dvipdf, if not using dvips). graphicx
  % will default to the driver specified in the system graphics.cfg if no
  % driver is specified.
  % \usepackage[dvips]{graphicx}
  % declare the path(s) where your graphic files are
  % \graphicspath{{../eps/}}
  % and their extensions so you won't have to specify these with
  % every instance of \includegraphics
  % \DeclareGraphicsExtensions{.eps}
\fi

% Document settings
\input{../config/settings}


\begin{document}

\message{ !name(sections/03-00-sim.tex) !offset(-11) }
\section{Simulaciones y experimentos}
    En esta sección se realizarán simulaciones y comparaciones entre distintos sistemas. Para demostrar la idea básica de sistema TIIR, se comenzerá examinando el sistema tal que

    \begin{equation}
      H(z) = \frac{z^{2}}{z^{2} - 1.9 \: z + 0.98} = \frac{B^{+}(z)}{A^{+}(z)}
    \end{equation}

    el cual se deseará truncar luego de un $N = 300 \: [\text{muestras}]$ arbitrario; esto es, se desea obtener una respuesta FIR de 301 pasos. Se comienza entonces formando el polinio de cancelación de cola como en la Eq. \ref{eq:25}, realizando división sintética de $z^{N} \: B(z)$  por $A(z)$, lo que resulta en el resto

    \begin{equation}
      B^{\prime +}(z) = - 0.162126 \: z + 0.13977
    \end{equation}

    De esta manera, se puede definir $H_{FIR}^{+}(z)$ según la Ec. \ref{eq:42}

    \begin{align}
      H_{FIR}^{+}(z) =& \frac{B(z) - z^{-N} \: B^{\prime + }(z)}{A(z)} \nonumber \\
      =& \frac{z^{2} + z^{-N} \: \left( - 0.162126 \: z + 0.139977 \right)}{z^{2} - 1.9 \: z + 0.98}
    \end{align}

    Entonces, se definen los coeficientes de $H_{FIR}^{+}(z)$  como

    \begin{itemize}
      \item $a = 0.162126$
      \item $b = 0.139977$
      \item $p = 1.9$
      \item $q = 0.98$
    \end{itemize}

    Así, se obtiene la FT de forma de llegar a un polinomio con el formato para graficar en GNU Octave

    \begin{align}
      H_{FIR}^{+}(z) =& \frac{z + a \: z^{-(N-1)} - b \: z^{-N}}{z^{2} - p \: z + q} \nonumber \\
      =& \frac{z^{2} + z^{-(N-1)} \: \left(a - b \: z^{-1} \right)}{z^{2} - p \: z + q} \nonumber \\
      =& \frac{z^{2} + \frac{a - b \: z^{-1}}{z^{N-1}}}{z^{2} - p \: z + q} \nonumber \\
      =& \frac{z^{N+1} + \left( a - b \: z^{-1} \right)}{z^{N-1} \: \left( z^{2} - p \: z + q \right)}
    \end{align}

    Ahora, si se desarrolla el término $b \: z^{-1}$

    \begin{align}
      H_{FIR}^{+}(z) =& \frac{z^{N+1} + a - \nicefrac{b}{z}}{z^{N-1} \: \left( z^{2} - p \: z + q \right)} \nonumber \\
      =& \frac{z^{N+2} + a \: z - b}{z^{N} \: \left( z^{2} - p \: z + q \right)} \nonumber \\
      =& \frac{z^{N+2} + a \: z - b}{z^{N+2} - p \: z^{N+1} + q \: z^{N}}
      \label{eq:03-00-hfir+}
    \end{align}

    Queda definida entonces la FT $H_{FIR}^{+}(z)$, y se puede insertar en GNU Octave con el algoritmo que se muestra en XXXX. En la Fig. \ref{fig:03-00-hfir+} se puede observar la respuesta al impulso de la misma.

    \begin{figure}
      \centering
      \includegraphics[width=0.52\textwidth]{../images/03-00-hfir+.png}
      \caption{Respuesta al impulso de $H_{FIR}^{+}$ según la Ec. \ref{eq:03-00-hfir+}}
      \label{fig:03-00-hfir+}
    \end{figure}

%%% Local Variables:
%%% mode: latex
%%% TeX-master: "../main"
%%% End:

\message{ !name(../main.tex) !offset(-53) }

\end{document}
