\message{ !name(../main.tex)}\documentclass[journal,transmag]{IEEEtran}

% Document preamble - loads packages
% *** MISC UTILITY PACKAGES ***
\usepackage[spanish]{babel}
\usepackage{amsmath, amsthm, amssymb, amsfonts} % Nicer mathematical typesetting
\usepackage{lipsum} % Creates dummy text lorem ipsum to showcase typsetting
\usepackage{siunitx}
\usepackage{nicefrac}

\usepackage{graphicx} % Allows the use of \begin{figure} and \includegraphics
\usepackage{float} % Useful for specifying the location of a figure ([H] for ex.)
\usepackage{caption} % Adds additional customization for (figure) captions
\usepackage{subcaption} % Needed to create sub-figures
\usepackage{tabularx} % Adds additional customization for tables
\usepackage{tabu} % Adds additional customization for tables
\usepackage{booktabs} % For generally nicer looking tables
\usepackage[nottoc,numbib]{tocbibind} % Automatically adds bibliography to ToC
\usepackage{hyperref} % Allows links and makes references and the ToC clickable
\usepackage[noabbrev, capitalise]{cleveref} % Easier referencing using \cref{<label>} instead of \ref{}

\usepackage{xcolor} % Predefines additional colors and allows user defined colors

% *** GRAPHICS RELATED PACKAGES ***
%
\ifCLASSINFOpdf
  % \usepackage[pdftex]{graphicx}
  % declare the path(s) where your graphic files are
  % \graphicspath{{../pdf/}{../jpeg/}}
  % and their extensions so you won't have to specify these with
  % every instance of \includegraphics
  % \DeclareGraphicsExtensions{.pdf,.jpeg,.png}
\else
  % or other class option (dvipsone, dvipdf, if not using dvips). graphicx
  % will default to the driver specified in the system graphics.cfg if no
  % driver is specified.
  % \usepackage[dvips]{graphicx}
  % declare the path(s) where your graphic files are
  % \graphicspath{{../eps/}}
  % and their extensions so you won't have to specify these with
  % every instance of \includegraphics
  % \DeclareGraphicsExtensions{.eps}
\fi

% Document settings
\input{../config/settings}


\begin{document}

\message{ !name(sections/02-01-fact-adit.tex) !offset(-11) }
\subsection{Método de diseño por factorización aditiva}
    Se vió que un filtro FIR tendrá fase linear si se cumple la Ec. \ref{eq:11}. Es posible mantener dicha relación utilizando filtros TIIR. Si $H_{FIR}^{+}$ es una función transferencia TIIR tal que:

    \begin{equation}
      H_{FIR}^{+}(z) = \sum_{k=0}^{N} {h_{k}^{+} \: z^{-k}} = \frac{B^{+}(z) - z^{-N} \: B^{+\prime} (z)}{A^{+} (z)}
      \label{eq:42}
    \end{equation}

    Formando la función transferencia de tiempo-contrario truncada:

    \begin{align}
      H_{FIR}^{-} (z) =& \sum_{k=0}^{N} {h_{k}^{-} \: z^{-k}} \\
      =& \sum_{k=0}^{N} {h_{k}^{+*} \: z^{k-N}} \\
      =& z^{-N} \left[ H_{FIR}^{ +} \left( \frac{1}{z^{*}} \right) \right] ^{*} \\
      =& \frac{z^{-N} \left[ B^{+} \left( \frac{1}{z^{*}} \right) - B^{+}' \left( \frac{1}{z^{*}} \right) \right] ^{*}}{\left[ A^{ + } \left( \frac{1}{z^{*}} \right) \right] ^{*}} \\
      =& \frac{-z \: B^{-\prime} (z) + z^{-N} \: B^{-} \: (z)}{A^{-}  (z)}
         \label{eq:47}
    \end{align}


%%% Local Variables:
%%% mode: latex
%%% TeX-master: "../main"
%%% End:

\message{ !name(../main.tex) !offset(-18) }

\end{document}
