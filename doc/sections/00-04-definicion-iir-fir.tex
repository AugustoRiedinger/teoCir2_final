\subsection{Definición de filtros FIR e IIR}
    Un filtro FIR causal de orden $N$ convencional se puede representar según:

    \begin{equation}
      y[n] = \sum_{k=0}^{N}{h_k x[n-k]}
    \end{equation}

    Donde la función de transferencia tiene la siguiente forma:

    \begin{equation}
      H_{FIR}(z) = h_0 + h_1 z^{-1} + \ldots + h_N z^{-N} = z^{-N} C(z)
    \end{equation}

    Y se define a $C(z)$ como el polinomio de orden $N$ formado por los coeficientes $h_k$.

    En cambio, un filtro IIR causal de orden $P$ tiene la relación:

    \begin{equation}
      y[n] = - \sum_{k=1}^{P}{a_k - y[n-k]} + \sum_{l=0}^{P}{b_l x[n-l]}
    \end{equation}

    Con la función transferencia correspondiente es:

    \begin{equation}
      H_{IIR}(z) = \frac{b_0 + b_1 z^{-1} + \ldots + b_P z^{-P}}{1 + a_1 z^{-1} + \ldots + a_P z^{-P}} = \frac{B(z)}{A(z)}
      \label{eq:6}
    \end{equation}

    Por tanto, $A(z)$ y $B(z)$ serán:

    \begin{equation}
      A(z) = z^{P} + a_1 z^{P-1} + \ldots + a_P
      \label{eq:8}
    \end{equation}

    \begin{equation}
      B(z) = b_0 z^P + b_1 z^{P-1} + \ldots + b_P
      \label{eq:9}
    \end{equation}

    Y pueden ser asumidos como los polinomios de grado $P$ en $z$.

    El retardo de grupo se define como:

    \begin{equation}
      \tau_d(\omega) = - \frac{d \: \arg\{H(e^{j\omega)}\}}{d\omega}
    \end{equation}

    El retardo de grupo a frecuencia normalizada $\omega=\nicefrac{2 \pi f}{f_s}$; donde $f_s$ es la frecuencia de muestreo; es el número de muestras con retraso experiezadas por la amplitud de la envolvente de banda ancha centrada en $\omega$.

    Un filtro de fase linear es aquel cuya respuesta en fase a una determinada frecuencia es una función lineal de la frecuencia; esto es, $\arg \{H(e^{j\omega})\}=K1\omega + K2$ para constantes $K1$ y $K2$. A partir de esta propiedad, se nota inmediatamente que el retardo de grupo es constante a todas las frecuencias. Filtro con una respuesta en fase lineal son usualmente los buscados, debido a que no poseen distorsión temporal dependiente de la frecuencia. Un filtro IIR con polos distintos de cero puede tener una fase lineal. Sin embargo, un filtro FIR con coeficiente $h_0,\ldots , h_N$ tiene fase lineal si existe un $\varphi$ tal que para todo $k \in \{0,\ldots , N\}$:

    \begin{equation}
      h_{N-k} = e^{j\varphi} h^{*}_k
      \label{eq:11}
    \end{equation}

    Esto es, si los coeficientes invertidos son conjugados complejos de la próxima secuencia sumados a una constante de cambio de fase, el retardo de grupo será entonces:

    \begin{equation}
      \tau_d (f) = \frac{N}{2}
    \end{equation}

%%% Local Variables:
%%% mode: latex
%%% TeX-master: "../main"
%%% End:
