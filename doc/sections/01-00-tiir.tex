\section{Filtros IIR truncados (TIIR)}
    Considere un filtro FIR con una secuencia geométricamente truncada $h_0,h_0p,\ldots , h_0 p^N$ como respuesta al impulso. Este filtro tiene la misma respuesta al impulso para los primeros $N+1$ términos que un filtro IIR de un solo polo con función de transferencia

    \begin{equation}
      H_{IIR}(z) = \frac{h_0}{1- p z^{-1}}
    \end{equation}

    Si se substrae la "cola" de la respuesta al impulso, se obtiene:

    \begin{equation}
      H_{FIR} = h_0 + h_0 p z^{-1} + \ldots + h_0 p^N z^{-N}
    \end{equation}

    \begin{equation}
      H_{FIR} = h_0 \frac{p^{N-1} z^{-(N+1)}}{1 - p z^{-1}}
      \label{eq:15}
    \end{equation}

    La recursión en el dominio del tiempo para este filtro estará dada como:

    \begin{equation}
      y[n] = \sum_{k=0}^{N}{h_0 p^k x[n-k]}
      \label{eq:16}
    \end{equation}

    \begin{equation}
      y[n] = p y[n-1] + h_0 \{x[n] - p^{N+1} x[n - (N+1)]\}
      \label{eq:17}
    \end{equation}

    Se nota que la primer formulación (Ec. \ref{eq:16}) requiere $N+1$ múltiplos y $N$ sumandos para ser implementada de forma directa; mientras que la segunda ecuación (Ec. \ref{eq:17}) requiere de solo tres múltiplos y dos sumandos, independientemente de $N$. Por tanto, se ve que, si se puede representar un FIR como una secuencia exponencial truncada, se encontraría una forma reducida de computar el filtro. Notar que el término $x[n - (N+1)]$ en la Ec. \ref{eq:17} todavía requiere un retardo para ser mantenido; y por tanto, no hay un reducción en cuanto a almacenamiento.

    Existe una cancelación de polo-cero en la representación dada en la Ec. \ref{eq:15}. Si $|p|<1$ no hay problemas, ya que el sistema será inherentemente estable. Si $|p|>1$, sin embargo, entonces existe un problema potencial debido al "modo oculto".

    La idea de esta sección funciona solo para casos en los que la multiplicidad de cada polo es uno; tal que cada modo exhiba un decaimiento exponencial simple.

    \subsection{Extensión a secuencias de alto orden}
    Se puede extender la idea ilustrada en la sección anterior para el caso de un polo simple con el objetivo de computar secuencias TIIR de cualquier denominador de $H(z)$. El procedimiento general es el de encontrar una la "cola" la función transferencia IIR:

    \begin{equation}
      H_{IIR}'(z) = h_0' z^{-1} + h_1' z^{-2} + \ldots = h_{N+1} z^{-1} + h_{N+2} z^{-2} + \ldots
    \end{equation}

    cuya respuesta al impulso, excepto por un cambio temporal de $N$ pasos, es igual a la "cola" de la función transferencia $H_{IIR}(z)$, la cual se pretende truncar luego del paso $N$.

    Entonces, multiplicando $H_{IIR}(z)$ por $z^N$ se obtiene:

    \begin{align}
      z^N H_{IIR} (z) =& h_0 z^N + \ldots + h_{N-1}z + h_N + h_{N+1} z^{-1} + \ldots \\
      =& C(z) + H_{IIR}'(z) \\
      =& \frac{z^N B(z)}{A(z)} \\
      =& C(z) + \frac{B'(z)}{A(z)}
    \end{align}

    donde $O[B'(z)] < O[A(z)] = P$. Se puede asumir que $O[B'(z)]=P-1$. $B'(z)$ es única y se puede obtener al realizar la división sintética de $z^N B(z)$ por $A(z)$ y encontrar el sobrante.

    Una vez obtenida $B'(z)$, se tiene que $H_{IIR}' = \nicefrac{B'(z)}{A(z)}$, y se puede escribir:

    \begin{equation}
      H_{FIR}(z) = H_{IIR}(z) - z^{-N} H_{IIR}'(z) = \frac{B(z)-z^{-N}B'(z)}{A(z)}
    \end{equation}

    El sistema correspondiente será entonces:

    \begin{equation}
      \begin{split}
        y[n] =& - \sum_{k=1}^{P}{a_k y[n-k]} + \sum_{l=0}^{P}{b_l x[n-l]}-\\
              & - \sum_{m=0}^{P-1}{b_m ' x[n-m-(N+1)]}
      \end{split}
      \label{eq:25}
    \end{equation}

    El hecho de que los denominadores de las funciones transferencia $H_{IIR}(z)$ y $H_{IIR}'(z)$ son los mismos permite un menor costo computacional debido al hecho que el IIR original y la \"cola\" IIR dinámica son las mismas y no se necesita realizar procedimientos duplicados. El costo computacional de este sistema IIR de orden $P$ es de $3P+1$ multiplicandos y $3P-2$ sumandos, independiente de $N$. Así, una ganancia computacional con esta clase de filtros FIR es conseguida si $N>3P$.

    El costo de almacenamiento para este filtro son $P$ muestras de salida para la realimentación dinámica IIR, $N$ muestras de entrada para el filtro FIR, y un adicional de $P$ muestras de entrada para la cancelación de la \"cola\"; significando $N+P$ muestras de entradas con retraso, de las cuales solo las últimas $P$ son utilizadas, y $P$ muestras de salida con retraso. Así, el algoritmo FIR requiere $2P$ más de almacenamiento que una implementación FIR directa.

%%% Local Variables:
%%% mode: latex
%%% TeX-master: "../main"
%%% End:

    \subsection{Otras arquitecturas}
    La implementación directa especificada según la Ec. \ref{eq:25} puede ser no deseada por varias razones. Por ejemplo, uno puede optar por utilizar una estructura factorizada tal como la forma bicuadrática cascada o la forma de fracciones paralelas parciales. Esta última está dada como:

    \begin{equation}
      H(z) = \sum_{k=1}^{N_p}{\sum_{l=1}^{M_k}{\frac{C_{k,l}}{(1-p_k z^{-1})^l}}}
      \label{eq:26}
    \end{equation}

    donde $N_p$ es el número de polos distintos, y $M_k$ es la multiplicidad del polo número $k$. El término $(k,l)$ es la expansión por fracciones parciales correspondiente al filtro con respuesta al impulso

    \begin{equation}
      h_{k,l}[n] = C_{k,l}
      \left(
      \begin{matrix}
               n + l - 1 \\
               n - 1 \\
      \end{matrix}
      \right)
           p_{k}^{n}
         \end{equation}

         Para formar el filtro TIIR, una cola del filtro IIR es derivada por cada fracción parcial utilizando división sintética como se demostró en la sección anterior. Cada respuesta TIIR es calculada de forma separada, y los resultados son sumados para formar la respuesta completa. La factorización no debe ser necesariamente completa como se muestra en la Ec. \ref{eq:26}. Es posible optar por un nivel intermedio de factorización, dejando algunos factores juntos y otros separados. Por ejemplo, se puede optar por agrupar los pares conjugados complejos para evitar aritmética compleja. Alternativamente, se puede dejar los terminos con los mismos polos juntos, según:

         \begin{equation}
           H(z) = \sum_{k=1}^{N_p}{\frac{B_k(z)}{(z-p_k)^{M_k}}}
         \end{equation}

         ya que para cacular la cola de la respuesta IIR para cada término de orden $n$ se necesita un polinomio de orden $n-1$ en el numerador de todas formas. La respuesta al impulso de  cada fracción parcial $k$ en este caso es:

         \begin{equation}
           h_{k}[n] = \sum_{l=0}^{M_k}{b_{k,l}}
           \left(
           \begin{matrix}
                    n + l + M_{k} -1 \\
                    M_{k} - 1 \\
            \end{matrix}
           \right)
                p_{k}^{n-l}
              \end{equation}

              donde $b_{k,l}=0,\ldots ,M_k$ son los coeficientes de $B_k(z)$.

              Otro ejemplo es agrupar a los factores estables juntos; esto es, aquellos con polos $p_k$ tales que $|p_k|<1$, e implemetar de forma separada polos inestables para los cuales $|p_k|>1$.

%%% Local Variables:
%%% mode: latex
%%% TeX-master: "../main"
%%% End:


%%% Local Variables:
%%% mode: latex
%%% TeX-master: "../main"
%%% End:
