\subsection{Comparación entre los distintos tipos de filtros}
    Los filtros IIR son ampliamente utilizados debido a su bajo costo computacional. Los filtros FIR, en cambio, permiten la posibilidad de implementar filtros lineales digitales con un retraso de grupo constante para todas las frecuencias. La contrapartida es que, para alcanzar funciones de transferencia de magnitud similar, los filtros FIR requieren un orden mucho mayor que su contraparte IIR. Por ejemplo,  La contrapartida es que, para alcanzar funciones de transfer    de magnitud similar, los filtros FIR requieren un orden mucho mayor que su contraparte IIR. Por ejemplo, un filtro general FIR de orden $N$ requieren $N+1$ múltiplos y $N$ sumandos.

    En ciertos casos, sin embargo, es posible diseñar filtros FIR con costo computacional comparable al de los filtros IIR mientras se mantienen las ventajas que proporcionan los filtros FIR. Este tipo de filtros FIR puede ser implementado de forma eficiente mediante secuencias truncadas de bajo orden de filtros IIR. A este tipo de filtros se los conoce como filtro TIIR.

%%% Local Variables:
%%% mode: latex
%%% TeX-master: "../main"
%%% End:
