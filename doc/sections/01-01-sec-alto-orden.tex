\subsection{Extensión a secuencias de alto orden}
    Se puede extender la idea ilustrada en la sección anterior para el caso de un polo simple con el objetivo de computar secuencias TIIR de cualquier denominador de $H(z)$. El procedimiento general es el de encontrar una la "cola" la función transferencia IIR:

    \begin{equation}
      H_{IIR}'(z) = h_0' z^{-1} + h_1' z^{-2} + \ldots = h_{N+1} z^{-1} + h_{N+2} z^{-2} + \ldots
    \end{equation}

    cuya respuesta al impulso, excepto por un cambio temporal de $N$ pasos, es igual a la "cola" de la función transferencia $H_{IIR}(z)$, la cual se pretende truncar luego del paso $N$.

    Entonces, multiplicando $H_{IIR}(z)$ por $z^N$ se obtiene:

    \begin{align}
      z^N H_{IIR} (z) =& h_0 z^N + \ldots + h_{N-1}z + h_N + h_{N+1} z^{-1} + \ldots \\
      =& C(z) + H_{IIR}'(z) \\
      =& \frac{z^N B(z)}{A(z)} \\
      =& C(z) + \frac{B'(z)}{A(z)}
    \end{align}

    donde $O[B'(z)] < O[A(z)] = P$. Se puede asumir que $O[B'(z)]=P-1$. $B'(z)$ es única y se puede obtener al realizar la división sintética de $z^N B(z)$ por $A(z)$ y encontrar el sobrante.

    Una vez obtenida $B'(z)$, se tiene que $H_{IIR}' = \nicefrac{B'(z)}{A(z)}$, y se puede escribir:

    \begin{equation}
      H_{FIR}(z) = H_{IIR}(z) - z^{-N} H_{IIR}'(z) = \frac{B(z)-z^{-N}B'(z)}{A(z)}
    \end{equation}

    El sistema correspondiente será entonces:

    \begin{equation}
      \begin{split}
        y[n] =& - \sum_{k=1}^{P}{a_k y[n-k]} + \sum_{l=0}^{P}{b_l x[n-l]}-\\
              & - \sum_{m=0}^{P-1}{b_m ' x[n-m-(N+1)]}
      \end{split}
      \label{eq:25}
    \end{equation}

    El hecho de que los denominadores de las funciones transferencia $H_{IIR}(z)$ y $H_{IIR}'(z)$ son los mismos permite un menor costo computacional debido al hecho que el IIR original y la \"cola\" IIR dinámica son las mismas y no se necesita realizar procedimientos duplicados. El costo computacional de este sistema IIR de orden $P$ es de $3P+1$ multiplicandos y $3P-2$ sumandos, independiente de $N$. Así, una ganancia computacional con esta clase de filtros FIR es conseguida si $N>3P$.

    El costo de almacenamiento para este filtro son $P$ muestras de salida para la realimentación dinámica IIR, $N$ muestras de entrada para el filtro FIR, y un adicional de $P$ muestras de entrada para la cancelación de la \"cola\"; significando $N+P$ muestras de entradas con retraso, de las cuales solo las últimas $P$ son utilizadas, y $P$ muestras de salida con retraso. Así, el algoritmo FIR requiere $2P$ más de almacenamiento que una implementación FIR directa.

%%% Local Variables:
%%% mode: latex
%%% TeX-master: "../main"
%%% End:
