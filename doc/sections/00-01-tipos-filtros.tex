\subsection{Tipos de filtros}

    Hay varios tipos de filtros así como distintas clasificaciones para estos filtros:

    \begin{itemize}
      \item De acuerdo con la parte del espectro que dejan pasar y atenúan hay:
            \begin{itemize}
              \item
              \item Filtros pasa alto.
              \item Filtros pasa bajo.
              \item Filtros pasa banda.
              \item Banda eliminada.
              \item Multibanda.
              \item Pasa todo.
              \item Resonador.
              \item Oscilador.
              \item Filtro peine (Comb filter).
              \item Filtro ranura o filtro rechaza banda (Notch filter).
            \end{itemize}
        \item De acuerdo con su orden:
            \begin{itemize}
              \item Primer orden.
              \item Segundo orden.
            \end{itemize}
        \item De acuerdo con el tipo de respuesta ante la entrada unitaria:
            \begin{itemize}
              \item FIR (Respuesta Finita al Impulso o /Finite Impulse Response/).
              \item IIR (Respuesta Infinita al Impulso o /Infinite Impulse Response/).
              \item TIIR (Respuesta Infinita Truncada al Impulso o /Truncated Infinite Impulse Response/).
            \end{itemize}
      \item De acuerdo con la estructura con que se implementa:
            \begin{itemize}
              \item Directa.
              \item Transpuesta.
              \item Cascada.
              \item Fase lineal.
              \item Laticce.
            \end{itemize}
    \end{itemize}

%%% Local Variables:
%%% mode: latex
%%% TeX-master: "../main"
%%% End:
