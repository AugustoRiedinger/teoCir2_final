\subsection{Método de diseño por factorización aditiva}
    Se vió que un filtro FIR tendrá fase linear si se cumple la Ec. \ref{eq:11}. Es posible mantener dicha relación utilizando filtros TIIR. Si $H_{FIR}^{+}$ es una función transferencia TIIR tal que:

    \begin{equation}
      H_{FIR}^{+}(z) = \sum_{k=0}^{N} {h_{k}^{+} \: z^{-k}} = \frac{B^{+}(z) - z^{-N} \: B^{+\prime} (z)}{A^{+} (z)}
      \label{eq:42}
    \end{equation}

    Formando la función transferencia de tiempo-contrario truncada:

    \begin{align}
      H_{FIR}^{-} (z) =& \sum_{k=0}^{N} {h_{k}^{-} \: z^{-k}} \\
      =& \sum_{k=0}^{N} {h_{k}^{+*} \: z^{k-N}} \\
      =& z^{-N} \left[ H_{FIR}^{ +} \left( \frac{1}{z^{*}} \right) \right] ^{*} \\
      =& \frac{z^{-N} \left[ B^{+} \left( \frac{1}{z^{*}} \right) - B^{+\prime} \left( \frac{1}{z^{*}} \right) \right] ^{*}}{\left[ A^{ + } \left( \frac{1}{z^{*}} \right) \right] ^{*}} \\
      =& \frac{-z \: B^{-\prime} (z) + z^{-N} \: B^{-} \: (z) }{A^{-} \: (z)}
      \label{eq:47}
    \end{align}

    donde los superíndices $+$ y $-$ denotan 'adelante' y filtro 'reverso-conjugado' respectivamente; y el superíndice $*$ denota conjugación compleja. De esta forma, comparando con las Ec. \ref{eq:8} y \ref{eq:9} se tiene:

    \begin{equation}
      A^{-}(z) = 1 + a^{*}_1 z + \ldots + a^{*}_P z^P
    \end{equation}

    \begin{equation}
      B^{-}(z) = b_0 + b^{*}_1 z + \ldots + b^{*}_P z^P
    \end{equation}

    Entonces:

    \begin{equation}
      B^{-\prime}(z) = b_0^{*\prime} + b_1^{*\prime} z + \ldots + b_{P_1}^{*\prime} z^{P-1}
    \end{equation}

    Se asume que $B^{+}(z)$ y $B^{-\prime}(z)$ tienen ordenes $P$ y $P-1$ respectivamente. Si se asume que $H_{FIR}^{ +}(z)$ es un filtro TIIR estable, entonces $H_{FIR}^{-}(z)$ es un filtro TIIR inestable cuyos modos escondidos son conjugados recíprocos de los de $H_{FIR}^{ +}(z)$.

    Utilizando un cambio de tiempo arbitrario $M \geq 0$ y cambio de fase $\varphi$, se define el filtro:

    \begin{align}
      H_{LPFIR}(z) =& H_{FIR}^{ + } (z) + e^{j\varphi} z^{-M} H_{FIR}^{-}(z) \\
      =& \sum_{k=0}^{N}{h_k^{ + } z^{-k}} + e^{j \varphi} \sum_{k=0}^{N}{h_k^{+*} z^{k-N-M}}
    \end{align}

    Se nota que este nuevo filtro FIR de longitud $M+N+1$ es invariante con respecto al orden reverso, conjugando los coeficientes, y multiplicando por un factor de fase $e^{j\varphi}$. De esta forma, se cumple la Ec. \ref{eq:11}; y así, $H_{LPFIR}(z)$ es un filtro FIR de fase lineal. Se puede ver la propiedad de fase lineal de forma más directa al notar primero que en el círculo unidad

    \begin{equation}
      H_{FIR}^{-}(e^{j\omega}) = e^{-jN\omega} \left[ H_{FIR}^{ + } \left( e^{j\omega} \right ) \right]^{*}
      \label{eq:53}
    \end{equation}

    tal que

    \begin{align}
      H_{LPFIR}(e^{j\omega}) =& H_{FIR}^{+}\left( e^{j\omega} \right) + e^{j\varphi} e^{-jM\omega} H_{FIR}^{-} \left( e^{j\omega} \right) \\
      =& e^{-j[(N + M)\omega + \varphi]/2} H_{FIR}^{ + } \left( e^{j\omega} \right)
    \end{align}

    por lo que el retardo de grupo resulta en:

    \begin{equation}
      \tau_d = \frac{M+N}{2}
    \end{equation}

    Es relativamente complicado diseñar un FFIR para un determinado set de especificaciones utilizando el método de factorización aditiva debido a que no es intuitivamente obvio como controlar la magnitud $H_{LPFIR}\left( e^{j\omega} \right)$. Sin embargo, para determinadas respuestas al impulso que están bien caracterizadas, ésta técnica provee una herramienta útil para el diseño de filtros FFIR.

%%% Local Variables:
%%% mode: latex
%%% TeX-master: "../main"
%%% End:
